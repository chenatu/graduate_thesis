
%%% Local Variables: 
%%% mode: latex
%%% TeX-master: t
%%% End: 

\chapter{总结与展望}

\section{论文工作总结}
本论文主要研究了基于命名数据网络的命名服务网络(Named Data Networking, NSN),及基于命名服务网络的存储服务。研究思路为,首先作者参加NDN社区开源项目设计了基于NDN的数据存储协议NDN Repo,并基于NDN Repo协议开发了数据存储软件repo-ng;以NDN Repo协议为基础,本文提出了一种基于NDN的一般化的服务网络设计框架,与web service,REST架构进行了比较研究。在新的命名服务网络架构基础上,重新规范了存储协议设计;在新的基于命名服务网络的存储协议之上,提出了一种分布式存储架构,以应用层帧设计理念为基础,减少了在网络与本地数据转换,以及网络传输中的冗余。

本文的主要结论与贡献为:
\begin{enumerate}
\item 本文首次提出了一种真正可以远程访问的命名数据网络的数据存储服务。以签名数据请求协议为基础,建立了远程存储主机的可靠访问模式。开发的repo-ng存储软件得到了NDN研究社区的广泛使用。
\item 以命名数据网络为基础,建立了从命名数据到命名服务的抽象。根据WSMF模型,提出了一套在命名服务网络基础上建立服务的设计原则。一方面,基于服务描述文档的服务架构可以提高服务网络的应用扩展性。另一方面,在网络层之上建立服务网络,简化底层网络协议栈,提高服务传输效率。本文对服务网络的扩展性,可用性以及传输效率进行了测试。在中等规模的网络中,NSN的存储服务可以在数据转发的性能瓶颈内进行线性扩展;局域网内服务迁移可以在1 ms左右恢复服务;与Web Service服务相比,提高了网络传输效率。
\item 以命名服务网络以及NDN Repo协议为基础建立了分布式的命名数据存储服务。命名服务网络构建了命名数据存储服务的操作性基础。同时根据应用层帧的设计原则,提出了本地转发数据请求表结构(LFIB)减少了网络数据与本地数据转换的冗余;提出了基于命名服务网络的underlay网络设计,减少了网络传输过程中,复杂网络协议栈的传输冗余。
\end{enumerate}

\section{未来的研究工作}
命名数据网络是一种新的面向数据的网络架构体系,在新体系下,需要特定协议支持上层的应用开发。本文的命名服务网络协议推出了一种支持安全可扩展的一种协议范式。未来可以作为IETF的一种基于NDN的上层协议进行更广泛的合作研究。在命名网络本身,如何构建更加可扩展的服务描述体系,如何更好的进行多服务的集成都是可以深入研究的问题。在应用层面,同样可以基于命名服务网络开发更广泛的服务应用促进研究。

在基于命名服务网络的存储方面,本论文提出LFIB结构以及underlay网络来提高数据处理与传输效率。但是在分布式存储中,数据一致性,服务可用性以及分区容错性都是可以进行深入研究的问题。多个元数据描述表如何协调,多点数据如何同步以及很多经典的分布式存储问题都是命名数据存储今后的研究方向。