
%%% Local Variables:
%%% mode: latex
%%% TeX-master: t
%%% End:
\secretlevel{绝密} \secretyear{2100}

\ctitle{基于命名机制的服务网络设计}
% 根据自己的情况选,不用这样复杂
\makeatletter
\ifthu@bachelor\relax\else
  \ifthu@doctor
    \cdegree{工学博士}
  \else
    \ifthu@master
      \cdegree{工学硕士}
    \fi
  \fi
\fi
\makeatother


\cdepartment[计算机]{计算机科学与技术系}
\cmajor{计算机科学与技术}
\cauthor{陈硕} 
\csupervisor{曹军威副研究院}
% 如果没有副指导老师或者联合指导老师,把下面两行相应的删除即可。
%\cassosupervisor{陈文光教授}
%\ccosupervisor{某某某教授}
% 日期自动生成,如果你要自己写就改这个cdate
%\cdate{\CJKdigits{\the\year}年\CJKnumber{\the\month}月}

% 博士后部分
% \cfirstdiscipline{计算机科学与技术}
% \cseconddiscipline{系统结构}
% \postdoctordate{2009年7月——2011年7月}

\etitle{An Design of Service Network Based on Naming mechanism} 
% 这块比较复杂,需要分情况讨论:
% 1. 学术型硕士
%    \edegree:必须为Master of Arts或Master of Science(注意大小写)
%              “哲学、文学、历史学、法学、教育学、艺术学门类,公共管理学科
%               填写Master of Arts,其它填写Master of Science”
%    \emajor:“获得一级学科授权的学科填写一级学科名称,其它填写二级学科名称”
% 2. 专业型硕士
%    \edegree:“填写专业学位英文名称全称”
%    \emajor:“工程硕士填写工程领域,其它专业学位不填写此项”
% 3. 学术型博士
%    \edegree:Doctor of Philosophy(注意大小写)
%    \emajor:“获得一级学科授权的学科填写一级学科名称,其它填写二级学科名称”
% 4. 专业型博士
%    \edegree:“填写专业学位英文名称全称”
%    \emajor:不填写此项
\edegree{Master of Engineering} 
\emajor{Control Science and Engineering} 
\eauthor{Chen Shuo} 
\esupervisor{Associate Professor Cao Junwei} 
%\eassosupervisor{Chen Wenguang} 
% 这个日期也会自动生成,你要改么?
% \edate{December, 2005}

% 定义中英文摘要和关键字
\begin{cabstract}
   近年来,互联网基础设施及应用服务快速发展,互联网“基础架构化”的趋势明显。在私有的分布式系统内部,服务多以服务接口的形式进行集成。在全局互联网中,越来越多的网络资源服务如Amazon云存储,微信公共账号等以SOAP(Simple Object Access protocol)或者REST(Representational State Transfer)接口的形式对外提供服务。而近些年云计算的发展与推广,使基于云服务尤其软件即服务(Software as a Service,SAAS)的接口服务得到大规模普及。
   SOAP和REST为当前最流行的两种Web Service服务形式。REST需要基于下层的HTTP协议。SOAP虽然对下层协议没有特定的限制,但是HTTP协议是最普遍的实现方式。HTTP协议也成为了事实上的网络瘦腰(thin waist)。而从HTTP协议开始,自上而下需要经过多层协议栈,而各层协议栈为了通用性没有对大规模的服务网络进行优化。HTTP协议以目标主机名为基础对数据包进行命名。近些年来,信息中心网络作为一种新的网络架构设计范式被提出来。其核心是以内容命名网络包来取代以地址命名的网络包。以其中的命名服务网络为例,以命名数据包替代IP网络包作为新的网络瘦腰。将服务网络与命名数据网络的性质有机结合起来,为服务网络从底层进行优化提供了新的研究方向。本文中,作者通过命名机制,提出一种重构面向服务的网络协议栈设计,并对该设计的协议性能,安全性等进行实验研究,最后提出一种将该服务网络应用于局部分布式系统的方案。作者的主要工作是:
  \begin{itemize}
    \item 开发一种基于命名服务网络的存储软件,并以此软件为原型提出基于命名机制的服务网络设计原则;
    \item 提出一种基于命名机制的服务网络协议设计即命名服务网络;
    \item 对命名服务网络原型进行实现,并对该网络服务进行多方位实验测试;
    \item 提出一种基于命名服务网络的服务设计方案。
  \end{itemize}
\end{cabstract}

\ckeywords{Web Service, 信息中心网络, 命名数据网络}

\begin{eabstract} 
   An abstract of a dissertation is a summary and extraction of research work
   and contributions. Included in an abstract should be description of research
   topic and research objective, brief introduction to methodology and research
   process, and summarization of conclusion and contributions of the
   research. An abstract should be characterized by independence and clarity and
   carry identical information with the dissertation. It should be such that the
   general idea and major contributions of the dissertation are conveyed without
   reading the dissertation. 

   An abstract should be concise and to the point. It is a misunderstanding to
   make an abstract an outline of the dissertation and words ``the first
   chapter'', ``the second chapter'' and the like should be avoided in the
   abstract.

   Key words are terms used in a dissertation for indexing, reflecting core
   information of the dissertation. An abstract may contain a maximum of 5 key
   words, with semi-colons used in between to separate one another.
\end{eabstract}

\ekeywords{\TeX, \LaTeX, CJK, template, thesis}
